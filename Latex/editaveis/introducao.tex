\chapter[Introdução]{Introdução}

\section{Contextualização e Justificativa}


A mineração consiste na extração, elaboração e beneficiamento de minerais que se encontram em
estado natural sólido, líquido ou gasoso.  Esta atividade está constantemente no dia a dia das pessoas, desde artigos em vidro (areia) e cerâmica (argila) até a fabricação de remédios, eletrônicos, etc. Segundo \citeonline{forooshani2013survey}, a indústria de mineração tem um papel vital na economia global. Em 2015, a capitalização de mercado estimada de empresas mineradoras globais é de cerca de \$962 bilhões dólares americanos. Uma grande parte destas operações são subterrâneas e envolve equipamentos e processos especializados. Os sistemas de comunicação desempenham um papel cada vez mais importante para garantir a segurança dos operários e otimizar o processo de mineração. O tamanho estimado do mercado de equipamentos de mineração subterrânea por si só foi calculado em cerca de \$45 milhões dólares em 2015, e uma parte pequena mas importante deste montante é atribuída a sistemas de comunicação

De acordo com \citeonline{barkand2006through}, os sistemas de comunicação atuais usados em minas subterrâneas empregam sistemas com fio. No entanto, estes sistemas de comunicação à cabo podem parar de operar como resultado dos incêndios, desabamentos ou explosões, ocorrentes em um desastre. Os sistemas de comunicação sem fio tem melhor probabilidade de sobreviver a explosões, desabamentos e inundações. Isso se dá pelo fato de ter pouca ou nenhuma dependência de um condutor sólido que deve permanecer intacto. Entretanto, os desafios técnicos tornam difícil estabelecer um sistema de comunicação sem fio prático em minas subterrâneas. 

A idéia de usar a terra como um meio comum para a comunicação volta a Nicola Tesla, já em 1899, através do uso de ondas com freqüências extremamente baixas (\textit{extremely low frequency })  \cite{wheeler1961radio}. Ainda segundo \citeonline{barkand2006through}, a maioria dos sistemas de comunicação sem fio utilizam topologias de comunicação em radiofrequência que exigem um caminho claro ou ao ar livre para a propagação do sinal, o que limita a comunicação para as entradas de minas adjacentes ou em torno dos pilares dos túneis. Durante emergências, o desabamento do telhado também pode bloquear ou limitar severamente a propagação de sinal de rádio convencional. No caso em que rochas (entre outros materiais de condutividade elétrica não desprezível) se tornam o meio de propagação, a alta atenuação de ondas eletromagnéticas em altas frequências não permite o uso desses sistemas supracitados. \cite{raab1995signal}

\clearpage

Utilizar sinais similares aos utilizados para transmissões em ar livre é inviável, pois esses sinais são incapazes de penetrar nas rochas (e demais materiais). No entanto, a atenuação de sinais eletromagnéticos em comunicações através da terra (\textit{Through-the-earth}) diminui com a frequência, e em frequências muito baixas (abaixo de 30kHz) é possível realizar uma comunicação entre o subterrâneo e a superfície diretamente. \cite{bandyopadhyay2010wireless}

\section{Objetivos}
\label{OBJETIVOS}
\subsection{Objetivos Gerais}

	Dentro deste contexto, o presente trabalho propõe a implementação de um sistema de comunicação de imagens em baixo consumo, mesmo com uma largura de banda de transmissão muito baixa, permitindo o envio de informações vitais por operários presos na mina para equipes de resgate. Para o cumprimento do objetivo geral, foram traçados alguns objetivos específicos citados abaixo.

\subsection{Objetivos Específicos}
\begin{itemize}
	\item[•] Estudo dos tipos de comunicação existentes atualmente nas minas subterrâneas a fim de identificar qual é o que mais se adapta às circunstancias exigidas (junho/2015);
	\item[•] Estudar técnicas rápidas de \textit{downsampling} para codificar a imagem em resolução mais baixa (julho/2015);
	\item[•] Desenvolvimento do algoritmo para estas mesmas técnicas;
	\item[•] Prototipagem de hardware e de software(agosto/2015);
	\item[•] Testes de hardware e de software(setembro/2015);
	\item[•] Redação do texto final para o trabalho de conclusão de curso(outubro/2015).
\end{itemize}