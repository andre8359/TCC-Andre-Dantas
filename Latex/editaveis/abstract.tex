\begin{resumo}[Abstract]
 \begin{otherlanguage*}{english}

Underground mining is one of the most extreme occupations on modern industries. First of all, mining operations are performed in very hazardous environments. Some of the main risks  are related to accidents that happen inside the mine, such as roof falls , fires, explosions and floods, that may hinder wired communication. For workers under these circumstances, wireless communication may offer the best chances of survival. This paper proposes the implementation of a type of low-power image encoding, which enables communication even with a  low-bandwidth, allowing the transmission of vital information by workers trapped in mines to rescue teams. The proposed communication system has a transmitter (that will be exposed to harsh conditions in the mine) that sends video sequences in mixed resolution to the receiver. On the other hand, the receiver has extensive computing resources and energy, as it will be located on the surface of the mine. The quality of the received video sequences is improved by a super-resolution technique based on example-based super-resolution. Tests were performed with JPEG and H.264 / AVC standards, varying the parameters that define the quantization coefficients of video encoding sequences and recording the qualities associated through PSNR and MSE metrics in order to evaluate the performance of the proposed algorithm for decoding on a practical situation. Reduction measures were obtained up to 75 \% for the H.264 standard and 65 \% for standard JPEG ,  these results were obtained especially for more static sequences, which was expected, and even for sequences with more movement and multiple entries of new elements in the scene. Super-resolved frames have higher quality than the interpolated ones. Thus, with these results, it was possible to fulfill the main objective of this work.
 
   \vspace{\onelineskip}
 
   \noindent 
   \textbf{Key-words}: Image Coding. Underground Mine Communications. Super-resolution. H.264. JPEG
    \end{otherlanguage*}
\end{resumo}
