\begin{resumo}[Abstract]
 \begin{otherlanguage*}{english}

Underground mining is one of the most extreme occupations from several perspectives. First of all, mining operations are carried out in very hazardous environments. Some of the main risks relate to accidents in the mine, such as roof falls , fires, explosions, floods, among others, these factors may hinder wired communiaction. For workers under these circumstances, wireless communications may offer the best chances of survival. This paper proposes the implementation of a low-power, low-bandwidth image communication system for underground mines, allowing the transmission of vital information by workers trapped in mines to rescue teams. The proposed system is based on interpolative image coding, where the input image's resolution is reduced before encoding, and recovered after decoding.
 
   \vspace{\onelineskip}
 
   \noindent 
   \textbf{Key-words}: Image Coding. Underground Mine Communications.  DCT. H.264. JPEG
    \end{otherlanguage*}
\end{resumo}
