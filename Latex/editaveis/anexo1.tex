\chapter[Fundamentação Teórica]{Fundamentação Teórica}

\section{Veículos Aéreos não Tripulados (VANTS)}
Os Veículos Aéreos Não Tripulados são aeronaves que possuem a capacidade de voo com controle a distância ou autônomo, sendo normalmente aplicadas em situações repetitivas, hostis, perigosas ou de difícil acesso para aeronaves convencionais \cite{Furtado}. 

Na última década, observou-se um grande esforço para desenvolvimento e emprego de VANTs, tanto para aplicações militares como civis. Os VANTs são comumente utilizados para fins militares, tais como o \textit{Global Hawk}, \textit{Aerosondem}, \textit{Predator} e \textit{GNAT750}, todos com autonomia de vôo superior a 40h. Em muitos casos, esses sistemas são importantes ferramentas táticas de apoio à manutenção da soberania.

Nas aplicações militares, conforme o Departamento de Defesa Norte Americano (DoD), observa-se um acentuado investimento a partir de 2001 nos Estados Unidos da América, conforme é visto na Figura \ref{PERFILANUAL}.

\begin{figure}
	\centering
	\includegraphics[keepaspectratio=true,scale=1.0]{figuras/PERFILANUAL.png}
    \caption{Perfil dos recursos anuais do DoD para VANTs.\citeonline{InstrumentacaodeAeronaves}}
    \label{PERFILANUAL}
\end{figure}

Em outro sentido, com menor estrutura e alcance mais limitado encontra-se os mini-VANTs. Estes são menores e requerem menor infraestrutura para realização de missões, podem ser operados por pequenas equipes. A área de mini-VANTs encontra forte potêncial de aplicação civil (Tabela \ref{tab02}). Nesta área, é possível observar a sua utilização em gerenciamento de queimadas, pesquisa ambiental, controle de poluição, segurança, monitoração de fronteira e agricultura \cite{OliveiraFA}. 

No Brasil existem importantes iniciativas civis de utilização dos VANTs, principalmente voltadas para as áreas agrícolas, gerenciamento de recursos e vigilância. Como exemplo, a Embrapa possui uma ação conjunta com a USP, denominado projeto ARARA, que busca desenvolver um VANT para monitoramento ambiental e agrícola \cite{RASIJR}. A Petrobrás investe nos VANTs para monitorar vazamentos e a Chesf para monitorar sua rede de transmissão de energia elétrica \cite{Furtado}.

\begin{table}
	\label{tab02}
    \begin{tabular}{ | p{9cm} | p{6cm} |}
    \hline
    Aplicação & Operador \\ \hline
    Vigilância Policial de Áreas Urbanas & Polícias Estaduais  \\ \hline
    Vigilância de Áreas de Fronteira & Polícia Federal  \\ \hline
    Inspeção de Oleodutos e Gasodutos & Petrobras  \\ \hline
    Controle de Safras Agrícolas & Embrapa  \\ \hline
    Levantamento de Recursos Florestais e 
    Controle de Queimadas & Ibama  \\ \hline
	Enlace de Comunicações & Empresa de Telecomunicações  \\ \hline
	Cobertura de Eventos para TV & Redes de TV  \\ \hline
    \hline
    \end{tabular}
    \caption{Iniciativas civis de utilização de VANTs no Brasil \cite{OliveiraFA}}
\end{table}

A capacidade de circular sem serem percebidos, guiados remotamente a partir de informação recebida por sensores e câmeras, faz com que os VANTs sejam motivo de desconfiança. Em vários países há debate sobre ética e moral no emprego de VANTs, principalmente no que se refere a questões de privacidade.

Há temor de que a falta de transparência no uso dos veículos encubra possíveis abusos no monitoramento de áreas e pessoas, com interceptação de conversas telefônicas, fotografias e filmagens feitas de maneira irregular. Países e indústrias já estão sujeitos à espionagem, por exemplo.

As autoridades temem ainda o risco de colisão com aviões e obstáculos aéreos, bem como a possibilidade de que o equipamento caia sobre áreas habitadas, colocando em risco a vida de pessoas em solo. Controlados de uma cabine, os VANTs circulam sem garantia de que os operadores tenham total conhecimento da situação no ar.

O caminho seguro para que VANTs consolidem-se como ferramentas benéficas à sociedade é a regulamentação e a fiscalização das aeronaves e de seus operadores.  A Agência Federal de Administração Aérea dos Estados Unidos (FAA), responsável pelo controle da aviação civil nos Estados Unidos deve divulgar ainda em 2013 normas referentes aos voos domésticos de VANTs. Vários países aguardam o documento para servir de base na criação de suas próprias leis. Atualmente, a operação civil ainda é bem controlada nos Estados Unidos, restrita a liberação de licenças individuais e proibida em regiões habitadas.

No Brasil, a Agência Nacional de Aviação Civil (Anac) reconhece a importância do uso civil dos VANTs, tanto para indústria como para a sociedade, mas afirma que, "devido aos novos desafios e características associadas ao voo remoto, são necessárias adequações na regulamentação deste tipo de aeronave para garantir níveis de segurança" \cite{PolemicosRevolucionarios}.

Tanto as normas da Anac quanto as regras do Departamento de Controle do Espaço Aéreo (Decea), da Aeronáutica, proíbem totalmente o voo de VANTs sobre cidades brasileiras. As demais operações precisam ser comunicadas à Aeronáutica com antecedência de 15 a 30 dias, para evitar que os veículos dividam o espaço aéreo com aviões comerciais.

Em outubro de 2012, a Anac publicou no Diário Oficial a Instrução Suplementar (IS) 21-002, que prevê requisitos básicos para certificar os veículos. VANTs totalmente autônomos são proibidos. Interessados em obter a licença devem enviar para a agência informações sobre o modelo e o propósito da operação.

O Brasil lidera na América do Sul, e também desponta no mundo, com iniciativas que envolvem o acesso de aeronaves remotamente controladas ao espaço aéreo. A Figura \ref{REQUISITOSANAC} apresenta a proposta da indústria à Anac para uso comercial.

\begin{figure}
	\centering
	\includegraphics[keepaspectratio=true,scale=0.7]{figuras/REQUISITOSANAC.png}
    \caption{Requisitos de uso do VANT propostos pela indústria à ANAC.}
    \label{REQUISITOSANAC}
\end{figure}

Os VANTs podem ser divididos em três principais grupos: os aviões, helicópteros/Multimotor e dirigíveis (\textit{blimps}). 

\subsection{Aviões}
Um exemplo de VANT é o Predator \textit{Unmanned Aerial Vehicle}, (Figura \ref{PREDATOR}) que foi desenvolvido pelo exército norte americano.

\begin{figure}
	\centering
	\includegraphics[keepaspectratio=true,scale=1.0]{figuras/PREDATOR.png}
    \caption{Predator UAV \cite{FMurtagh}.}
    \label{PREDATOR}
\end{figure}

No Brasil também há o uso de VANTs por parte da força aérea, a Figura \ref{COMPARACAOVANTS} mostra o comparativos de dois VANTs, um da Força Aérea Brasileira (FAB) e outro da Polícia Federal (PF).

\begin{figure}
	\centering
	\includegraphics[keepaspectratio=true,scale=1.0]{figuras/COMPARACAOVANTS.png}
    \caption{Comparação entre dois VANTs utilizados no Brasil.}
    \label{COMPARACAOVANTS}
\end{figure}

Uma desvantagem em relação aos multimores é que os aviões precisam de uma pista de pouso e decolagem, portanto não podem pousar nem decolar na vertical. Por outro lado o avião possui mais estabilidade ao atingir a velocidade mínima de sustentação podendo então planar, pois a força de sustentação na aeronave fica em suas asas (Figura \ref{FORCASAVIAO}). Devido sua forma de locomoção e velocidade, os aviões são usados em ambientes externos (\textit{outdoor}).

\begin{figure}
	\centering
	\includegraphics[keepaspectratio=true,scale=0.8]{figuras/FORCASAVIAO.png}
    \caption{Forças atuando na asa de um avião \cite{FisicaCotidiano}.}
    \label{FORCASAVIAO}
\end{figure}

\subsection{Dirigível (\textit{blimps})}
O dirigível não precisa de propulsores para gerar sustentação aerodinâmica, limitando o uso destes dispositivos apenas para se locomover. Embora, dentre os três modelos apresentados, este tenha uma maior estabilidade, seja silencioso e mais fácil de controlar, são mais volumosos. Portanto, limita-se o seu uso a publicidade em eventos, estampado alguma propaganda no seu corpo (normalmente preenchido com gás hélio, por ser mais leve que o ar), principalmente em eventos realizados em locais fechados (Figura \ref{DIRIGIVEL}).

\begin{figure}
	\centering
	\includegraphics[keepaspectratio=true,scale=0.7]{figuras/DIRIGIVEL.png}
    \caption{Dirigível usado para propagandas. \cite{NDalalBTriggs}.}
    \label{DIRIGIVEL}
\end{figure}

\subsection{Helicópteros (Multimotor)}
O primeiro quad-rotor surgiu em 1922 (Figura \ref{QUADROTOR}), foi construído para o serviço aéreo militar dos Estados Unidos pelo George de Bothezat, mas o projeto foi esquecido pela dificuldade e complexidade que o piloto tinha para controlar a aeronave. No entanto, com o avanço da tecnologia e a estabilidade sendo controlada de forma automática, tal plataforma volta a ganhar destaque.

\begin{figure}
	\centering
	\includegraphics[keepaspectratio=true,scale=0.7]{figuras/QUADROTOR.png}
    \caption{De Bothezat Quadrotor, 1923. \cite{Quadcopter}.}
    \label{QUADROTOR}
\end{figure}

O multimotor é uma plataforma que, diferente do avião, possui sua força de sustentação nas asas, a força de sustentação é dividida em múltiplos rotores de mesma potência, de forma que o torque de reações de um cancele o do outro. Com isso, uma das virtudes do multimotor é que ele possui seis graus de liberdade, sendo eles translação ao longo nos três eixos X, Y e Z e rotação em torno destes mesmos eixos: \textit{pitch}, \textit{roll} e \textit{yaw} (Figura \ref{ESQUEMAQUADRIMOTOR}).

\begin{figure}
	\centering
	\includegraphics[keepaspectratio=true,scale=0.7]{figuras/ESQUEMAQUADRIMOTOR.png}
    \caption{Esquema físico de quadrimotor. \cite{MOzuysal}.}
    \label{ESQUEMAQUADRIMOTOR}
\end{figure}

Na Tabela \ref{tab03} são mostradas as velocidade de cada motor para realizar os devidos movimentos de rotação nos eixos do quad-rotor.

\begin{table}
	\label{tab03}
	\centering
    \begin{tabular}{ | c | c | c | c | c |}
    \hline
    Movimento & Motor 1 & Motor 2 & Motor 3 & Motor4 \\ \hline
    para cima (throttle+) & $v_1$ + $\varDelta$v & $v_2$ + $\varDelta$v & $v_3$ + $\varDelta$v & $v_4$ + $\varDelta$v \\ \hline
    para baixo (throttle-) & $v_1$ - $\varDelta$v & $v_2$ - $\varDelta$v & $v_3$ - $\varDelta$v & $v_4$ - $\varDelta$v \\ \hline
    para frente (pitch+) & $v_1$ - $\varDelta$v & $v_2$ + $\varDelta$v & $v_3$ & $v_4$ \\ \hline
    para trás (pitch-) & $v_1$ + $\varDelta$v & $v_2$ - $\varDelta$v & $v_3$ & $v_4$ \\ \hline
    para direita (roll+) & $v_1$ & $v_2$ & $v_3$ + $\varDelta$v & $v_4$ - $\varDelta$v \\ \hline
    para esquerda (roll-) & $v_1$ & $v_2$ & $v_3$ - $\varDelta$v & $v_4$ + $\varDelta$v \\ \hline
    horário (yaw+) & $v_1$ + $\varDelta$v & $v_2$ + $\varDelta$v & $v_3$ - $\varDelta$v & $v_4$ - $\varDelta$v \\ \hline
    anti-horário (yaw-) & $v_1$ - $\varDelta$v & $v_2$ - $\varDelta$v & $v_3$ + $\varDelta$v & $v_4$ + $\varDelta$v \\
    \hline
    \end{tabular}
    \caption{Rotação ao longo dos eixos da aeronave.}
\end{table}

Para o projeto optou-se pelo uso de um multi-rotor para realizar os experimentos, pois esse pode permanecer de forma estática para captura de imagens. A Figura \ref{ESQUELEQUADRIMOTOR} mostra quad-rotor utilizado.

\begin{figure}
	\centering
	\includegraphics[keepaspectratio=true,scale=0.8]{figuras/ESQUELEQUADRIMOTOR.png}
    \caption{Esqueleto de um quadrimotor utilizado no prmeiro experimento deste trabalho.}
    \label{ESQUELEQUADRIMOTOR}
\end{figure}

Os componentes utilizados para montar o quad-rotor são:

\begin{itemize}
\item Motor A2212 - 930Kv;
\item \textit{ESC30A BEC}; 
\item Controladora \textit{Crius AIOne} e Hélices 4.5 x 10. 
\item Transmissor e receptor de vídeo (FPV)

\end{itemize}

\section{Sistemas Embarcados}
Nos primeiros anos dos computadores digitais na década de 1940, eles eram dedicados a uma única tarefa. Eram, entretanto, muito grandes para serem considerados embarcados. O primeiro sistema embarcado reconhecido mundialmente foi o \textit{Apollo Guidance Computer}, desenvolvido nos EUA por Charles Stark Draper no MIT para a NASA. O "computador de guia", que operava em tempo real, era considerado o item eletrônico mais arriscado do projeto \textit{Apollo}. No projeto desenvolvido pelo MIT foram usados circuitos integrados monolíticos para reduzir o tamanho e peso do equipamento e aumentar a sua confiabilidade \citeonline{ELETRICA}.

Os sistemas embarcados estão mudando a forma como as pessoas vivem, trabalham, estudam, divertem e se interagem. Exemplos de tais sistemas (Figura \ref{ABSCAMERA}) são os \textit{smartphones}, \textit{MP3 player}, o sistema de controle dos automóveis (computador de bordo, sistema ABS), os computadores portáteis, os fornos de microondas com controle de temperatura inteligente, as máquinas de lavar e outros eletrodomésticos.

\begin{figure}
	\centering
	\includegraphics[keepaspectratio=true,scale=0.8]{figuras/ABSCAMERA.png}
    \caption{Sistema ABS e câmera digital \citeonline{ABOUTAUTO}.}
    \label{ABSCAMERA}
\end{figure}

Um sistema embarcado é um sistema microprocessado no qual o computador é completamente encapsulado ou dedicado ao dispositivo ou sistema que ele controla. Diferentemente de computadores de propósito geral, como o computador pessoal, um sistema embarcado realiza um conjunto de tarefas pré-definidas, geralmente com requisitos específicos. Já que o sistema é dedicado a tarefas específicas, através de engenharia pode-se otimizar o projeto, reduzindo tamanho, recursos computacionais e custo do produto. 

Sistemas embarcados utilizam vários tipos de processadores: DSPs (\textit{digital signal processors} - processadores digitais de sinais), microcontroladores, microprocessadores. Ao contrário do mercado de computadores pessoais, que á basicamente dominado pelos processadores de arquitetura x86 da \textit{Intel/AMD}, sistemas embarcados utilizam amplamente as arquiteturas \textit{ARM, PowerPC, PIC, AVR, 8051, Coldfire, TMS320, blackfin}.

Uma possibilidade de uso de uma plataforma embarcada para o projeto seria o \textit{Raspberry Pi}. Este é considerado um "computador", que tem o tamanho de um cartão de crédito, é desenvolvido no Reino Unido pela Fundação Raspberry Pi. O hardware é uma única placa. Seu principal objetivo é estimular o ensino de ciência da computação básica em escolas. Segundo o fabricante \textit{Farnell Newark} o computador é baseado em um chip \textit{BroadcomBCM2835}, que dispõe de um \textit{processador ARM1176JZF-Sde} 700 MHz, \textit{GPUVideoCore} IV, e 512 Megabytes de memória RAM \cite{DRossJLimRSLinMHang}.

\begin{figure}
	\centering
	\includegraphics[keepaspectratio=true,scale=0.8]{figuras/RASPI.png}
    \caption{Raspberry Pi.}
    \label{RASPI}
\end{figure}

Os motivos que levaram à escolha desta placa (Figura \ref{RASPI}) de desenvolvimento aliado ao baixo custo, foram as dimensões reduzidas (86,5mm x 56mm), interface \textit{Ethernet}, USB, HDMI (que a maioria dos concorrentes da \textit{Raspberry Pi} ainda não disponibiliza) e \textit{slot} para cartão SD. Nessa aplicação esse cartão é utilizado para embarcar o Linux, sistema operacional escolhido para a utilização do software no processamento de imagens. No entanto após realizar testes do algoritmo neste dispositivo, foi possível concluir que ainda serão necessárias otimizações no algoritmo para torna-lo funcional, pois a captura de vídeo neste dispositivo se mostrou ineficiente no desempenho de rastreamento. A otimização do algoritmo fugiria bastante da proposta deste trabalho e pode ser realizada em trabalhos futuros. Devido a isso se utilizou uma plataforma PC para realizar os testes do algoritmo. 

\section{Processamento de Imagem}
Processamento de imagem é qualquer forma de processamento de dados no qual a entrada e saída são imagens, tais como fotografias ou quadros de vídeo. O interesse em métodos de processamento de imagens digitais vem aumentando devido a duas principais áreas de aplicação:

\begin{enumerate}
\item Melhoria da informação de imagens e vídeos para interpretação humana;
\item Processamento de dados de imagem para o armazenamento, transmissão e representação para a percepção de máquina autônomas.
\end{enumerate}

Uma das primeiras aplicações da primeira categoria remonta ao começo deste século, onde buscavam-se formas de aprimorar a qualidade de impressão de imagens digitalizadas transmitidas através do sistema \textit{Bartlane}, que transmitia imagens por cabo submarino entre Londres e Nova Iorque. Os primeiros sistemas \textit{Bartlane}, no início da década de 20, codificavam uma imagem em cinco níveis de intensidade distintas. Esta capacidade seria expandida, já em 1929, para 15 níveis, ao mesmo tempo em que era desenvolvido um método aprimorado de revelação de filmes através de feixes de luz modulados por uma fita que continha informações codificadas sobre a imagem \cite{MARQUESFILHO}.

Os elementos de um sistema de processamento de imagens de uso genérico são mostrados na Figura \ref{ELEMENTOSPI}. Este diagrama \cite{MARQUESFILHO} permite representar desde sistemas de baixo custo até sofisticadas estações de trabalho utilizadas em aplicações que envolvem intenso uso de imagens. Ele abrange as principais operações que se pode efetuar sobre uma imagem, a saber: aquisição, armazenamento, processamento e exibição.

\begin{figure}
	\centering
	\includegraphics[keepaspectratio=true,scale=0.7]{figuras/ELEMENTOSPI.png}
    \caption{Elementos de um sistema de processamento de imagens \cite{InstrumentacaodeAeronaves}.}
    \label{ELEMENTOSPI}
\end{figure}

Aliado com o processamento de imagens digital, geralmente também tem-se a visão computacional, ou seja, um sistema computadorizado capaz de adquirir, processar e interpretar imagens correspondentes a cenas reais onde, a partir das informações, um sistema pode tomar decisões. A Figura \ref{ELEMENTOSPI} mostra esquematicamente o sistema de visão computacional. 

\section{Proposta do Sistema}

\begin{figure}
	\centering
	\includegraphics[keepaspectratio=true,scale=0.7]{figuras/DIAGRAMAPROPOSTO.png}
    \caption{Diagrama de blocos do sistema proposto.}
    \label{DIAGRAMAPROPOSTO}
\end{figure}

Nesse trabalho, levantou-se os principais blocos (Figura \ref{DIAGRAMAPROPOSTO}) que compõem o sistema proposto para a solução do problema. O sistema é composto por oito partes, mas nesse trabalho o sistema foi reorganizado em três seções, sendo que a uma seção é referente a plataforma VANT, que pode ser um multi-rotor ou uma asa fixa que agrupa os blocos 1, 2, 3, 4 e 5, conforme é visto o agrupamento na Figura \ref{DIAGRAMAPROPOSTO}, a segunda seção referente ao sistema de captura e o local onde as informações serão processadas são os blocos 6 e 7 e a terceira seção, algoritmo OpenTLD é representada pelo bloco 8.

\subsection{Plataforma VANT}
Nessa seção será aparesentada os blocos que são referentes a plataforma VANT,  A Figura \ref{DIAGRAMAVANT} representa diagrama de blocos elétrico-eletrônio da plataforma VANT.

\begin{figure}
	\centering
	\includegraphics[keepaspectratio=true,scale=0.7]{figuras/DIAGRAMAVANT.png}
    \caption{Diagrama de blocos elétrico-eletrônico de uma plataforma VANT.}
    \label{DIAGRAMAVANT}
\end{figure}

Em aeromodelismo, para se controlar uma aeronave ou embarcação remotamente, utiliza- se um link de RF (Rádio Frequência) do tipo FM (\textit{Frequence Modulation}) composto por um rádio transmissor e um rádio receptor, cuja portadora pode ser de 72MHz ou 2,4GHz, onde tais frequências são liberadas pelo orgão competente (ANATEL no Brasil) para este fim. Os comandos de voo são enviados serialmente, de 20ms em 20ms, e modulados por posição de pulso, isto é, no formato PPM (\textit{Pulse Position Modulation}). O segundo bloco que constitui uma plataforma VANT é o rádio receptor, resposável por demultiplexar o sinal e enviar para a placa controladora a nformação enviada pelo rádio transmissor.

O terceiro bloco é referente a placa microcontroladora. A placa principal, ou microncotrolada, deve ser responsável por:

\begin{itemize}
\item capturar os sinais oriundos do sensores: acelerômetros (X, Y e Z),  giroscópios (X, Y e Z) e sistema de navegação (GPS);
\item capturar os canais (comandos de voo \textit{throttle}, \textit{pitch}, \textit{roll} e \textit{yaw}) do rádio receptor;
\item rodar algum algoritmo de controle para estabilização de voo;
\item gerar os sinais de PWM para os ESCs;
\end{itemize}

Uma controladora que pode ser utilizada é a ardupilot. A Figura \ref{ARDUPILOT} mostra a microcontroladora ardupilot.

\begin{figure}
	\centering
	\includegraphics[keepaspectratio=true,scale=0.7]{figuras/ARDUPILOT.png}
    \caption{Microcontroladora Ardupilot.}
    \label{ARDUPILOT}
\end{figure}

Essa microcontroladora pode ser utilizada para o uso de asas fixas ou multi-rotores, sendo que ela é capaz de controlar até oito motores, conforme é visto na Figura \ref{PLATAFORMASARDUPILOT}.

\begin{figure}
	\centering
	\includegraphics[keepaspectratio=true,scale=0.8]{figuras/PLATAFORMASARDUPILOT.png}
    \caption{Plataformas controladas pela ardupilot.}
    \label{PLATAFORMASARDUPILOT}
\end{figure}

O quarto bloco é referente a instrumentação ou seja a placa sensores. É nesse bloco onde se encontra o acalerometro, os giroscópios e o sistema de navegação (GPS), para proporcionar uma melhor estabilização da aeronave. O quinto bloco é referente aos ESCs e os motores do tipo \textit{brushless}.

Diferentemente dos motores de corrente contínua (DC), os motores do tipo brushless, não são alimentados através de escovas, mas sim através de um circuito eletrônico. Este circuito, além de prover energia aos enrolamentos de um motor \textit{brushless} a partir de uma fonte DC, também realiza um controle de velocidade em malha fechada. Por isto, tal circuito é chamado de Controle Cletrônico de Velocidade, ESC (\textit{Electronic Speed Control}).

Um ESC é basicamente dividido em duas partes: uma de controle e outra de potência. A primeira, é onde a controladora ESC recebe o sinal PWM com período de 20ms, vindo da microcontroladora, por exemplo a Ardupilot e gera três outros, normalmente trapezoidais e defasados entre si de \ang{120}. Esse bloco pode variavar em quantidade ESCs e motores, dependendo da plataforma a ser utilizada. 

Outras plataformas multi-rotor já são usadas pela polícia norte americana e do Canadá é a plataforma de uso comercial o \textit{Draganflyer X6} Figura \ref{DRAGANFLYER}. A plataforma o \textit{Draganflyer X6} é em fibra de carbono, contém 11 sensores internos (três giroscópios, três acelerômetros, três magnetômetros, um barômetro e um receptor GPS), possui retorno de vídeo ao vivo. Até a data 21/10/2013 o custo era de aproximadamente 15 mil dólares.

\begin{figure}
	\centering
	\includegraphics[keepaspectratio=true,scale=0.8]{figuras/DRAGANFLYER.png}
    \caption{Aeronave Draganflyer X6.}
    \label{DRAGANFLYER}
\end{figure}

Outra plataforma também utilizada pela polícia americana é o AeroVironment Qube drone Figura \ref{QUBEDRONOE}.  A autonomia da aeronave é superior a 40 minutos, link de dados bidirecional com 1 km e criptografado e o sistema é capaz de rastreamento de movimento. 

\begin{figure}
	\centering
	\includegraphics[keepaspectratio=true,scale=0.6]{figuras/QUBEDRONOE.png}
    \caption{Aeronave AeroVironment Qube drone.}
    \label{QUBEDRONOE}
\end{figure}

\subsection{Sistema de Captura e Processamento das Informações}
\textcolor{red}{Comentar que o sistema de captura foi feito por FPV- placa de captura, depois tentou uma wifi com a GoPro.
Falar que primeiramente tentou com a raspberry rodar o código, depois com a beaglebone e depois no computador, concluiu que o algoritmo nao esta otimizado para rodar em ambiente embarcado, por isso se escolheu o PC.}

\subsection{Algoritmo \textit{OpenTLD}}
Neste trabalho foi utilizado o algoritmo de Tracking-Learning-Detection (TLD) proposto por Kalal et al. \cite{KalalMikolajczykMatas}. Ele foi primeiramente disponibilizado em MATLAB e utilizado para rastrear objetos em vídeos. O algoritmo usado neste trabalho foi proposto por Georg Nebehay \cite{NebehayGeorg}, posteriormente implementado em C++ e publicado sobre os termos da GNU com o nome de OpenTLD.

	Esse algoritmo foca no rastreamento de um único alvo.  A abordagem desse é baseada no fluxo ótico usado no método de Lucas e Kanade \cite{LucasKanade}. Para a detecção de objeto, é seguido o método de detector de objetos durante rastreamentos instáveis \cite{ZKalalJMatasKMikolajczykOnline} e são mantidos os modelos que são normalizados em brilho e tamanho. 
	
Os modelos de exemplos positivos do objeto e exemplos negativos encontrados no plano de fundo são mantidos separados. Esses modelos formam a  base de um detector de objetos, que é executado independentemente do rastreador. Novos modelos são adquiridos usando a proposta de aprendizado de positivos e negativos \cite{ZKalalJMatasKMikolajczykPN}. Se o detector encontra um local de uma imagem com uma alta similaridade com um dos modelos, o rastreador é reinicializado neste local. Visto que a comparação de modelos é muito dispendiosa computacionalmente, é aplicada uma abordagem de cascata para a detecção de objetos. 

No aprendizado de modelos positivos e negativos, são usados um classificador de fern \cite{MOzuysal} baseado em padrões binários de 2-bits e um modelo único fixo. A detecção de objeto em cascata consiste em um detector de primeiro plano, um filtro de variância, um classificador de fern aleatório \cite{VLepetitPLaggerPFua} e um método comparador de modelos. Em contraste a Kalal et al., esse método não considera a deformação de imagens para o aprendizado. A Figura \ref{DIAGRAMATLD} apresenta o fluxo usado nessa abordagem. A inicialização leva ao passo de aprendizagem. No próximo passo, o rastreador e o detector são executados em paralelo e seus resultados são unidos em um único resultado final. Se esse resultado passar do estágio de validação, o aprendizado é realizado. Então o processo se repete. 

Na Figura \ref{DIAGRAMATLD} o processo de rastreamento é iniciado pela seleção manual do objeto de interesse, não sendo mais requerida a interferência do usuário.

\begin{figure}
	\centering
	\includegraphics[keepaspectratio=true,scale=0.6]{figuras/DIAGRAMATLD.png}
    \caption{Diagrama do TLD.\cite{NebehayGeorg}}
    \label{DIAGRAMATLD}
\end{figure}

\textcolor{red}{Nos anexos de I a VII, detalhes do funcionamento do algoritmo TDL e os códigos principais são apresentados.}

