\begin{resumo}

A mineração subterrânea é uma das atividades mais extremas da indústria moderna, sendo realizada em ambientes relativamente perigosos. Alguns dos principais riscos estão relacionados a acidentes dentro da mina, como desabamentos, incêndios, explosões e inundações, que podem inviabilizar comunicações com fio. Para trabalhadores sujeitos a estas circunstâncias, a comunicação sem fio pode oferecer as melhores chances de sobrevivência. O presente trabalho propõe a implementação de um sistema de comunicação de imagens em baixa largura de banda e baixo consumo, permitindo o envio de informações vitais por operários presos em minas para equipes de resgate. O sistema proposto é baseado na codificação interpolativa de imagens, aonde a resolução da imagem de entrada é reduzida antes da codificação, e recuperada após a decodificação. 


\vspace{\onelineskip}
    
 \noindent
 \textbf{Palavras-chave}: Codificação de Imagens. Comunicação para Minas Subterrâneas. H.264. JPEG 
\end{resumo}
