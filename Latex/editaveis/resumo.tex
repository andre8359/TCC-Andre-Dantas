\begin{resumo}

A mineração subterrânea é uma das atividades mais extremas da indústria moderna, sendo realizada em ambientes relativamente perigosos. Alguns dos principais riscos estão relacionados a acidentes dentro da mina, como desabamentos, incêndios, explosões e inundações, que podem inviabilizar comunicações com fio. Para trabalhadores sujeitos a estas circunstâncias, a  comunicação sem fio pode oferecer as melhores chances de sobrevivência. O presente trabalho consiste na implementação de um tipo de codificação de imagens em baixo consumo, que viabiliza a comunicação mesmo com uma largura de banda de transmissão muito baixa, permitindo o envio de informações vitais por operários presos na mina para equipes de resgate. Foi proposto um sistema de comunicação em que o transmissor (que  estará exposto às condições adversas de dentro da mina) envia sequências de vídeo em resolução mista para o receptor. Este, por sua vez, terá amplos recursos de computação e energia, já que ele estará na superfície da mina. A qualidade das sequências de vídeo recebidas é melhorada através de uma técnica de super-resolução baseada na super-resolução por exemplos. Foram realizados testes com os padrões JPEG e H.264/AVC, em que se variam os parâmetros que definem os coeficientes de quantização na codificação das sequências de vídeo e registrou-se as qualidades associadas  através das métricas PSNR(\textit{Peak Signal to Noise Ratio}) e MSE(\textit{Medium Square Error}), a fim de avaliar o desempenho do algoritmo proposto para a decodificação em uma situação prática. Foram obtidas medidas de redução de até 75\% para o padrão H.264 e 65\% para o padrão JPEG em relação a codificação com a resolução original, estes resultados foram obtidos especialmente para sequências mais estáticas, como esperado. Para sequências com muito movimento e várias entradas de novos elementos nas cenas os resultados obtidos foram mais modestos. No entanto,  os quadros super-resolvidos continuam possuindo maior qualidade que os interpolados. Assim, a implementação pode ser utilizada no desenvolvimento futuro do sistema proposto.

\vspace{\onelineskip}
    
 \noindent
 \textbf{Palavras-chave}: Codificação de Imagens. Comunicação para Minas Subterrâneas. Super-resolução. H.264. JPEG. 
\end{resumo}
