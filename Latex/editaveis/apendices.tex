\begin{apendicesenv}

\partapendices

\chapter{Detalhamento da tecnologia utiliziada}
\label{apen:tec}

    Segue o detalhamento das tecnologias utilizadas no protótipo do transmissor e do receptor.

\section{Transmissor}

Utilizou-se \textit{Raspberry Pi2 Model B}, rodando um sistema operacional \textit{Arch Linux Arm}, em um cartão micro SD de 8 Gb,  com um \textit{Python} versão 2.7.
Utilizou-se o editor de texto \textit{Vim}, assim como conceitos de orientação a objetos e o \textit{multimedia framework} FFmpeg versão 2.8.2, fazendo uso principalmente do \textit{FFmpeg Tools}, uma ferramente que permite redimensionar, codificar e converter videos de diversos formatos e um \textit{framework} de versionamento de código Git versão 2.1.4 .


\section{Receptor}

    Já no receptor utilizou-se um Notebook Asus X44C-VX029R - Intel Core i3-2330M - RAM 4GB - HD 320GB, rodando um sistema operacional \textit{Debian} 8 (Jiessie), com um \textit{GNU Octave}, versão 3.8.2.
Utilizou-se \textit{multimedia framework} FFmpeg versão 2.8.2, fazendo uso principalmente do \textit{FFmpeg Tools}, uma ferramente que permite redimensionar, codificar e converter videos de diversos formatos e um \textit{framework} de versionamento de código Git versão 2.1.4 . 




\end{apendicesenv}
