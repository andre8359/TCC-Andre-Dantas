\chapter[Considerações Preliminares]{Considerações Preliminares}

Em função da demanda de sistemas de comunicações subterrâneas sem fio, onde a largura de banda é estreita, o redimensionamento aparece como uma boa alternativa para a transmissão de imagens , reduzindo a informação a ser enviada assim como a banda ocupada por ela. Após efetuado todo o levantamento de referências bibliográficas, foi possível compreender a complexidade da implementação do redimensionamento, assim como da codificação da imagem, utilizando padrões difundidos em todo o mundo (JPEG e H.264).


Ao longo desse trabalho, apresentou-se pontos chave das operações de codificação e redimensionamento. Abordou-se informações sobre o processamento de imagens digitais como um todo, com uma atenção especial na DCT que é uma ferramenta utilizada tanto no redimensionamento quanto na codificação. Discutiu-se também as principais características dos padrões de codificação JPEG e H.264.

\section{Cronograma}

O cronograma para as atividades que serão realizadas para a continuação deste trabalho é apresentado na tabela \ref{cronograma}. As atividades planejadas são:
\begin{enumerate}

	\item Estudo dos tipos de comunicação existentes atualmente nas minas subterrâneas a fim de identificar qual é o que mais se adapta às circunstancias exigidas;
	\item  Estudar técnicas rápidas de \textit{downsampling} para codificar a imagem em resolução mais baixa;
	\item  Desenvolvimento do algoritmo para estas mesmas técnicas;
	\item  Prototipagem de hardware e de software;
	\item  Testes de hardware e de software;
	\item  Redação do texto final para o trabalho de conclusão de curso;
	\item Apresentação do trabalho de conclusão de curso.
\end{enumerate}

\newpage

\begin{table}[h]
    \centering
    \begin{tabular}{|c|c|c|c|c|c|c|c|}
        \hline
        \textbf{Ativ.} & \textbf{Jun 2015}&\textbf{Jul 2015} & \textbf{Ago 2015} & \textbf{Set 2015}
        & \textbf{Out 2015} & \textbf{Nov 2015} & \textbf{Dez 2015} \\
        \hline\hline

        \hline
        1   & •& • &  &   &   &   &   \\

        \hline
        2   &•& • &   &&   &   &   \\

        \hline
        3   &   &  • & • & &   &   &   \\

        \hline
        4   &  &   & • &• & &   &   \\

        \hline
        5   & &  &   & • &•  &   &   \\

        \hline
        6   &  & &   &   &  & • &   \\

        \hline
        7   &   &&   &   & &  & •  \\
        \hline
    \end{tabular}

    \caption{Cronograma de atividades para TCC2}
    \label{cronograma}
\end{table}
